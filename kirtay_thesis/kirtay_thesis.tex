% Author: Murat Kirtay, Robotics Laboratory
% Description: M.Sc. thesis


\documentclass[a4,12pt]{ozu-thesis}


%\usepackage{amsmath,amssymb,latexsym,float,epsfig,subfigure}
%\usepackage{xcolor}
%%\documentclass[conference]{IEEEtran}
\usepackage{amssymb,amsmath,epsfig}
%%\usepackage{easyturk}
%%\usepackage[latin5]{inputenc}
%\usepackage{subfigure}
\usepackage{graphicx}
%\usepackage{graphics}
%\usepackage{tabularx,caption}
%\usepackage{url}
%\usepackage{subfigure}

\usepackage{amsmath,amsfonts,amsthm,amssymb}
%\usepackage{setspace}
\usepackage{subfigure}
\usepackage{fancyhdr}
\usepackage{lastpage}
\usepackage{extramarks}
\usepackage{chngpage}
\usepackage{soul}
\usepackage[usenames,dvipsnames]{color}
\usepackage{graphicx,float}
\usepackage{epstopdf}
\usepackage{ifthen}
\usepackage{listings}
\usepackage{courier}
%\usepackage{subfig}
\usepackage{epigraph}
\usepackage{float}
\usepackage[]{algorithm2e}
\usepackage{multirow, tabularx}
\usepackage{colortbl}
\usepackage[table]{xcolor}


\title{Computational Approaches to Modeling Brain Mechanisms} %% If you want to specify a linebreak
                               %% in the thesis title, you MUST use
                               %% \protect\\ instead of \\, as \\ is a
                               %% fragile command that \MakeUpperCase
                               %% will break!
\author{Murat K{\i}rtay}
\department{Department of Computer Science}

%% By default, this makes a M.Sc. thesis cover. If you are writing a Ph.D. dissertation uncomment
%% the following.
%% \dissertationtrue

%% If you are writing a thesis proposal, uncomment the following.
%% \thesisproposaltrue

%% THe Committee Members are to be listed in the form
%%
%%  \member{Name}[Department][Institution]
%%
%% The second and third arguments are optional, but if you wish to
%% supply the third, you must supply the second. Department defaults
%% to the department defined above and Institution defaults to Ozyegin University


% The First Member is your Advisor. 
\firstmember{Professor Erhan \"{O}ztop, Advisor}

% The next member is your second advisor - if exists. Otherwise, it will be your external examiner.
\secondmember{Professor Bar\i\c{s} Aktemur}

% Include the other members of your committee. 
\thirdmember{Professor Alper A\c{c}{\i}k}[Department of Psychology]
%[Bo\u{g}azi\c{c}i University]
\fourthmember{Professor Hasan S\"{o}zer}
\fifthmember{Professor Furkan K{\i}ra\c{c}}[Department of Computer Science]



\degree{Master of Science}
\copyrightyear{2015}
\submitdate{June 2015} % Must be the month and year of graduation,
                         % not thesis approval! 
                         
%% The date the committee members sign the thesis form. Usually this is your defense date. Printed
%% on the approval page.
\approveddate{..... ..... 2015}

% You need to specify the .bib files here. You may have multiple files. Write their names separated by commas.}

\bibfiles{example-thesis}

%% The following are the defaults
%%    \titlepagetrue
%%    \signaturepagetrue
%%    \copyrighttrue
%%    \figurespagetrue
%%    \tablespagetrue
%%    \contentspagetrue
%%    \dedicationheadingfalse
%%    \bibpagetrue
%%    \thesisproposalfalse
%%    \strictmarginstrue
%%   \dissertationfalse
%%    \listmajorfalse

%%%%%%%%%%%%%%%%%%%%%%%%%%%%%%%%%%%%%%%%%%%%%%%%%%%%%%%%%%%%%%%%%%%%%%%%%%%%%%
% Below lines are collected from various online resources 
% check misc.tex
\newcommand{\todo}[1]{\textcolor{red}{TODO: #1}\PackageWarning{TODO:}{#1!}}
%%%%%%%%%%%%%%%%%%%%%%%%%%%%%%%%%%%%%%%%%%%%%%%%%%%%%%%%%%%%%%%%%%%%%%%%%%%%%%

\begin{document}

% We will use the standard IEEE transactions bibliography style in the thesis format. 
\bibliographystyle{ieeetr}
%%

\begin{preliminary}

%%
\begin{dedication}
\null\vfil
{\large
\begin{center}
To Non-fraud Academicians, Singularitarians, Cosmists, Hackers and AGI Researchers
\\
\todo{Never change this part}

\end{center}}
\vfil\null
\end{dedication}
%%

%%
\begin{abstract}
\input{../chapters/abstract/abstract.tex}

\end{abstract}
%%

%%
\begin{ozetce}
\input{../chapters/abstract/ozetce.tex}

\end{ozetce}
%%

%%
\begin{acknowledgements}
\input{../chapters/acknowledgement/acknowledgement2.tex}

\end{acknowledgements}
%%


% print table of contents, figures and tables here.
\contents
%
% If you need a "List of Symbols or Abbreviations" look into
% ozu-thesis-gloss.sty.

\end{preliminary}
%%

\chapter{Introduction}

\input{../chapters/introduction/introduction.tex}

\section{Motivation and Background}
\input{../chapters/introduction/motivation.tex}


\subsection{Existing Works on Mirror Neurons}
\input{../chapters/introduction/mn_lit_review.tex}

\subsection{Existing Works on Emotions}

\input{../chapters/introduction/emotion_lit_review.tex}
\newpage
\section{Contributions}

\input{../chapters/introduction/contributions.tex}

\section{Thesis Organization}
\input{../chapters/introduction/organization.tex}

%---------MIRROR NEURONS----------
\chapter{Mirror Neurons}

\input{../chapters/mirror_neuron/mirror_neuron_intro.tex}

\section{Experiments and Neural Data Set Features}

\input{../chapters/mirror_neuron/exp_unit_set.tex}

\subsection{Experiment conditions}
\input{../chapters/mirror_neuron/exp_conds.tex}

\subsection{Neural data set specifications}
\input{../chapters/mirror_neuron/unit_set.tex}

\section{Evaluation Metrics}
\input{../chapters/mirror_neuron/eval_mat.tex}

\section{Methods}
\subsection{Linear regression based decoding}

\input{../chapters/mirror_neuron/linear_reg.tex}

\subsubsection{Single unit linear regression}
\input{../chapters/mirror_neuron/su_linear_reg.tex}

\subsubsection{Multi unit linear regression}
\input{../chapters/mirror_neuron/mu_linear_reg.tex}

\subsection{Cross validation}
\input{../chapters/mirror_neuron/cross_val.tex}

%\subsubsection{Single unit cross validation}
%\subsubsection{Multi unit cross validation}
%\subsection{Cross decoding among events}
%\subsubsection{Observation to execution}
%\subsubsection{Execution to Observation}

%\subsection{Time interval decoding}
%\input{../chapters/mirror_neuron/time_interval.tex}

\section{Results and Discussions}
\input{../chapters/mirror_neuron/results.tex}

\subsection{Mirror neuron detection methods}
\subsubsection{Execution vs. observation decoding}
\input{../chapters/mirror_neuron/exe_obs_results.tex}

\subsubsection{Event-wise cross decoding}
\input{../chapters/mirror_neuron/cross_decoding.tex}

%\input{../chapters/mirror_neuron/exe_obs_results1.tex}
%\input{../chapters/mirror_neuron/exe_obs_results2.tex}

%\subsection{Temporal mirror neuron}
%\section{Conclusions}

%%-----------EMOTION------------------
\chapter{Emotions}
%\section{Motivation}
\input{../chapters/emotion/motivation.tex}

%\section{Existing Work on Emotions}
%\input{../chapters/emotion/lit_review.tex}

\section{Biological Background}
\input{../chapters/emotion/bio_back.tex}

\section{Methods}
\input{../chapters/emotion/method.tex}
\subsection{Behaviour trough self regulation of the Hopfield Neural Network}
\input{../chapters/emotion/hopfield.tex}
\subsection{Behaviour trough energy regulation}
\input{../chapters/emotion/energy_reg.tex}

\section{Experimental Setup}
\input{../chapters/emotion/exp_setup.tex}
\subsection{Hardware setup and data flow}
\input{../chapters/emotion/hw_setup.tex}
\subsection{Darwin-OP and behavior execution process}
\input{../chapters/emotion/exe_proc.tex}

\section{Results and Discussions}
\input{../chapters/emotion/results.tex}
\subsection{Analysis of energy conservation}
\input{../chapters/emotion/result_energy_con.tex}
\subsection{Emotion based network dynamics}
\input{../chapters/emotion/result_network_dyn.tex}


\chapter{Conclusions}


\input{../chapters/conclusion/conclusions.tex}


\section{Mirror Neurons}
\input{../chapters/conclusion/conclusions_mn.tex}
\section{Emotions}
\input{../chapters/conclusion/conclusions_em.tex}




\begin{thebibliography}{99}
\input{../chapters/references/references.tex}

\end{thebibliography}

%\chapter{Conclusion}

%\nocite{*}
%% We need this since this file doesn't ACTUALLY \cite anything...
%%
\appendix
%\chapter{Some Ancillary Stuff}



\begin{postliminary}
%\references
%\begin{vita}
%You write your vita here.
%\end{vita}
\end{postliminary}

\end{document}
