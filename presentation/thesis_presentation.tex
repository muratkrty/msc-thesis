%%%%%%%%%%%%%%%%%%%%%%%%%%%%%%%%%%%%%%%%%
% Beamer Presentation
% LaTeX Template
% Version 1.0 (10/11/12)
%
% This template has been downloaded from:
% http://www.LaTeXTemplates.com
%
% License:
% CC BY-NC-SA 3.0 (http://creativecommons.org/licenses/by-nc-sa/3.0/)
%
%%%%%%%%%%%%%%%%%%%%%%%%%%%%%%%%%%%%%%%%%

%------------------------------------------------------------------------------
%	PACKAGES AND THEMES
%------------------------------------------------------------------------------

\documentclass{beamer}

\mode<presentation> {

% The Beamer class comes with a number of default slide themes
% which change the colors and layouts of slides. Below this is a list
% of all the themes, uncomment each in turn to see what they look like.

%\usetheme{default}
%\usetheme{AnnArbor}
%\usetheme{Antibes}
%\usetheme{Bergen}
%\usetheme{Berkeley}
%\usetheme{Berlin}
%\usetheme{Boadilla}
%\usetheme{CambridgeUS}
%\usetheme{Copenhagen}
%\usetheme{Darmstadt}
%\usetheme{Dresden}
%\usetheme{Frankfurt}
%\usetheme{Goettingen}
%\usetheme{Hannover}
%\usetheme{Ilmenau}
%\usetheme{JuanLesPins}
%\usetheme{Luebeck}

\usetheme{Madrid}
%\usetheme{Malmoe}
%\usetheme{Marburg}
%\usetheme{Montpellier}
%\usetheme{PaloAlto}
%\usetheme{Pittsburgh}
%\usetheme{Rochester}
%\usetheme{Singapore}
%\usetheme{Szeged}
%\usetheme{Warsaw}

% As well as themes, the Beamer class has a number of color themes
% for any slide theme. Uncomment each of these in turn to see how it
% changes the colors of your current slide theme.

%\usecolortheme{albatross}
%\usecolortheme{beaver}
%\usecolortheme{beetle}
%\usecolortheme{crane}
%\usecolortheme{dolphin}
%\usecolortheme{dove}
%\usecolortheme{fly}
%\usecolortheme{lily}
%\usecolortheme{orchid}
%\usecolortheme{rose}
%\usecolortheme{seagull}
%\usecolortheme{seahorse}
%\usecolortheme{whale}
%\usecolortheme{wolverine}

%\setbeamertemplate{footline} % To remove the footer line in all slides uncomment this line
%\setbeamertemplate{footline}[page number] % To replace the footer line in all slides with a simple slide count uncomment this line

%\setbeamertemplate{navigation symbols}{} % To remove the navigation symbols from the bottom of all slides uncomment this line
}

\usepackage{graphicx} % Allows including images
\usepackage{booktabs} % Allows the use of \toprule, \midrule and \bottomrule i
\usepackage{courier}
\usepackage{subfig}
\usepackage{float}


\usepackage{multirow, tabularx}


\usepackage{epstopdf}



\definecolor{darkgreen}{rgb}{0, .6, 0}
%-----------------------------------------------------------------------------
%	TITLE PAGE
%-----------------------------------------------------------------------------

\title[ Mirror Neurons and Emotions]{Computational Approaches to Brain Mechanisms of Action Recognition and Emotion
 } % The short title appears at the bottom of 

\author{Murat K{\i}rtay } % Your name
\institute[] % Your institution as it will appear 
{ \\ % Your institution for the title page
\medskip
\textit{muratkrty@gmail.com} % Your email address
}
\date{\today} % Date, can be changed to a custom date

\begin{document}

\begin{frame}
\titlepage % Print the title page as the first slide
\end{frame}



%---------------------------------------------
%	PRESENTATION SLIDES
%---------------------------------------------
%\section{SCM and VCS ?}
%%------------------------------------------------
%
%\begin{frame}
%\frametitle{Why SCM/VCS tools?}
%
%\begin{center}
%\framebox{ \includegraphics[height=2in]{figures/git.jpg}}
%\footnote{http://git-scm.com/book/en/Getting-Started-Git-Basics} 
%\end{center}
%\end{frame}
%------------------------------------------------


%------------------------------------------------
%------------------------------------------------
\section{Motivation \#1} % Sections can be
%------------------------------------------------
%------------------------------------------------

\begin{frame}
\frametitle{Motivation \#1}

\begin{block}{Why Brain Mechanisms?}
\begin{itemize}
\item Target source for solving general intelligence
\item Achieved only once (human level intelligence) 
\item Everything is computable \tiny(Demis Hassabis) 
\end{itemize} 
\end{block}

\begin{block}{Why Computational Approaches?}
\begin{itemize}
\item To gain an understanding of brain mechanisms
\item To predict the outcome of natural processes 

\end{itemize} 
\end{block}

\begin{block}{Why Mirror Neurons and Emotions?}
\begin{itemize}
\item Polarized ideas
\item Potential impacts 

\end{itemize} 
\end{block}

\end{frame}

%------------------------------------------------
\section{ Motivation \#2 }
%------------------------------------------------

\begin{frame}

\begin{center}
\frametitle{Motivation \#2}
\framebox{ \includegraphics[height=2.5in]{figures/marr2.png}}

\end{center}
\end{frame}

%------------------------------------------------

%----EMOTIONS

\section{ What is emotion?}
%------------------------------------------------
\begin{frame}
\frametitle{What is emotion? }

\begin{block}{What is emotion?}
No universal definition.\\
``Everyone knows what it is until they are asked to define it." 
\end{block}

\textbf{Major functions of the emotions;}
\begin{itemize}

\item Facilitate the recall of memories
\item Trigger the motivated behaviors

\item \textcolor{darkgreen}{Short-cut cognitive processes}
\item Considered as source of reinforcement
\item Communication: big data can be transmitted (scream)
\item Constant change in endocrine system
\item ...
\end{itemize}

\end{frame}
%------------------------------------------------

\section{Hypothesis \& Biological Background}
%------------------------------------------------

\begin{frame}
\frametitle{Hypothesis}
\begin{block}{Hypothesis \& Biological Background}
The computational short-cut mechanisms on cognitive processes to facilitate 
energy economy give rise to what we call emotion.
\end{block}
Is this approach \textcolor{darkgreen}{biologically realistic}?
\begin{columns}[l|c]
  \column{2.9in}  % slides are 3in high by 5in wide

\begin{itemize}\scriptsize
\item Human brain accounts for \textbf{20\%} $O_{2}$
\item \textbf{10\%} of a resting humans energy is used to \\keep brain's batteries charged
\item \textbf{Blowfly:} transmission of a \textbf{bit} at a chemical\\ synapse cost $10^{4}$ ATP
\begin{block}{}\scriptsize
If metabolic energy is limiting, then neurons, neural codes and neural circuits will have evolved to reduce metabolic demands [1].

\end{block}
\end{itemize}
  		
  		
\column{1.8in}
\framebox{ \includegraphics[height=1.2in]{figures/vision.eps}}

\end{columns}
\footnote{\tiny [1] Laughlin et al. The metabolic cost of neural information. }
 
\end{frame}

%------------------------------------------------

%------------------------------------------------

\section{Method \& Experiments}
%------------------------------------------------

\begin{frame}
\frametitle{Method \& Experiments}



\begin{block}{}\scriptsize
\begin{itemize}
\item The PC or robot’s camera sees one pattern at a time 
\item Hopfield Network tries to recall it as one of the learned patterns
\item No interaction with operator
\item No predefined emotions
\end{itemize}

\end{block}


\begin{columns}[l|c]
  \column{2.9in}  % slides are 3in high by 5in wide

\begin{itemize}
\item \textcolor{darkgreen}{\textbf{Offline experiments}}
\begin{enumerate}\scriptsize
\item Understand the behavior of Hopfield Network
\item Facilitate threshold derivation
\item Post-processing experiment data for evaluation
\end{enumerate}

\item \textcolor{darkgreen}{\textbf{Online robot experiments}}
\begin{enumerate}\scriptsize
\item Realization of online experiment in Darwin-OP
\item Observing internal dynamics leads to
non-deterministic behavior execution

\end{enumerate}

\end{itemize}
  		
  		
\column{1.8in}
\framebox{ \includegraphics[height=1.6in]{figures/darwinSit.eps}}

\end{columns}

 
\end{frame}

%------------------------------------------------

\section{Offline PC experiment}
%------------------------------------------------

\begin{frame}
\frametitle{Offline PC experiment}
\begin{center}

\begin{tabular}{|@{}l|@{}l|}
  \hline
  
 \includegraphics[scale=0.3]{figures/combinedPatterns1.eps} (a)
& \includegraphics[scale=0.3]{figures/combinedPatterns2.eps} (b)\\ \hline

  \hline
\end{tabular}

\end{center}

\begin{itemize}\scriptsize
\item The convergence time and number of bits flipped versus contamination rate 
was investigated. Each contamination level repeated 20 times.
\item Fig. (a) shows the obtained convergence time versus contamination level 
for each learned pattern. Fig. (b) show number of bits flips versus contamination 
level for each pattern.
\end{itemize}

\begin{block}{}\scriptsize
\centering \textcolor{darkgreen}{\textbf{Threshold derivation}}

\end{block}

\end{frame}
%------------------------------------------------
%------------------------------------------------

\section{Online robot experiment}
%------------------------------------------------

\begin{frame}
\frametitle{Online robot experiment}
\begin{center}
\framebox{ \includegraphics[height=2.3in]{figures/baseStation.eps}}
\end{center}

\end{frame}
%------------------------------------------------


\section{Experiment DEMO}
%------------------------------------------------

\begin{frame}
\frametitle{DEMO}
%\begin{center}
%\framebox{ \includegraphics[height=2.9in]{figures/media.eps}}
%\end{center}

\begin{center}


\huge \textcolor{darkgreen}{\textbf{DEMO}}\\
\tiny \textbf{``Simulation is doomed to succeed." (R. Brooks) }
\end{center}

\end{frame}
%------------------------------------------------
\section{Results}
%------------------------------------------------

\begin{frame}
\frametitle{Results}



\begin{block}{Results}\scriptsize
\begin{itemize}
\item Biologically plausible model proposed and implemented.
\item A model emergence of higher level emotions trough neuro-computational
energy regulation is explored.
\item The model is realized on Darwin-OP (as opposed to being simulation study)

\end{itemize}

\end{block}




\begin{columns}[l|c]
  \column{2.9in}  % slides are 3in high by 5in wide

\begin{itemize}
\item \textcolor{darkgreen}{\textbf{Remaining works}}
\begin{enumerate}\scriptsize
\item Bio-inspired framework for emotion
\item Modeling brain regions: PFC and Amygdala
\item Objective evaluation of robot behavior
\item Integration with complex cognitive architecture
\item Foraging theory and swarm system
\end{enumerate}



\end{itemize}
  		
  		
\column{1.8in}
\begin{block}{}
\textcolor{darkgreen}{The continuation of this work may lead to a novel interdisciplinary research 
field for emotion, cognitive robotics, and computational neuroscience.}
\end{block}

\end{columns}

 
\end{frame}
%-----------------------

\section{Brain Mechanism: Mirror Neuron }
%------------------------------------------------
\begin{frame}
\frametitle{Mirror Neuron}

\centering \huge \textbf{Mirror Neuron}

\end{frame}
%------------------------------------------------

%------------------------------------------------
%------------------------------------------------
\section{ Mirror Neurons }
%------------------------------------------------

\begin{frame}
\frametitle{ Mirror Neurons}


\centering \textcolor{darkgreen}{\textbf{Fires in both execution and observation conditions.}}
\centering
\framebox{ \includegraphics[height=1.8in, width=4in]{figures/mirror.eps}}

%\textcolor{darkgreen}{\textbf{Tablolar{\i}n de\u{g}erlendirilmesi; }}
\footnote{\tiny  VI. Mirror Neuron Systems in Monkey, Rizzolatti et al.}

\begin{block}{Mirror Neurons}
``mirror neurons will do for psychology what DNA did for biology: they will provide a unifying framework ..."
\end{block}

\end{frame}
%------------------------------------------------
\section{ Problem Definition }
%------------------------------------------------
\begin{frame}
\frametitle{Problem Definition}

\begin{center}
\textcolor{darkgreen}{\textbf{Extracting mirror neuron candidates among
available data set in a mechanistic way.}}
\end{center}

\textbf{Performed methods on data:}
\begin{enumerate}
\item Linear Regression
\item Threshold based classification
\item Binary Regression
\item LOO Cross-Validation
\item Correlation Analysis
\item Canonical Correlation Analysis
\item ...

\end{enumerate}

All proposed methods applied on both single unit and multi unit sets.




\end{frame}
%------------------------------------------------
\section{ Experimental Setup and Conditions}
%------------------------------------------------

\begin{frame}

\begin{center}
\frametitle{Experimental Setup and Conditions}
\centering \textbf{Objects}\\
\framebox{ \includegraphics[height=1.2in]{figures/objects1.eps}}\\
\centering \textbf{Conditions}\\
\framebox{ \includegraphics[height=1.25in]{figures/experiments.eps}}
\end{center}
\end{frame}
%------------------------------------------------
%------------------------------------------------
%\section{ Data Structure }
%%------------------------------------------------
%
%\begin{frame}
%
%\begin{center}
%\frametitle{Data Structure}
%\textcolor{darkgreen}{\textbf{Good Unit}}\\
%\framebox{ \includegraphics[height=1.1in, width=3.6in]{figures/unit10.eps}}\\
%\textbf{Not Good Unit}\\
%\framebox{ \includegraphics[height=1in, width=3.6in]{figures/unit13.eps}}
%
%\end{center}
%\end{frame}



\section{ Linear Regression}
%------------------------------------------------

\begin{frame}
\frametitle{Regression and Cross-Validation }
\begin{center}


\begin{equation}\label{eq:linearReg}
    X*W=Y
\end{equation}


\begin{equation}\label{eq:multiUnit}
\begin{pmatrix}
 $Unit10-Obj1-r1$ & $Unit19-Obj1-r1$ \\ 
$Unit10-Obj2-r1$ & $Unit19-Obj2-r1$\\ 
$Unit10-Obj3-r1$ & $Unit19-Obj3-r1$\\ 
$Unit10-Obj4-r1$ & $Unit19-Obj4-r1$\\ 
  \vdots &\vdots\\ 
$Unit10-Obj1-r10$ & $Unit19-Obj1-r10$ \\ 
$Unit10-Obj2-r10$ & $Unit19-Obj2-r10$\\ 
$Unit10-Obj3-r10$ & $Unit19-Obj3-r10$\\ 
$Unit10-Obj4-r10$ & $Unit19-Obj4-r10$\\ 

\end{pmatrix}
\times
W
=
\begin{pmatrix}
 1\\ 
2\\ 
3\\ 
4\\ 
  \vdots\\ 

 1\\ 
2\\ 
3\\ 
4\\ 
\end{pmatrix}
\end{equation}
\begin{equation}\label{eq:predOut}
         X*W=Y_{pred}
\end{equation}
\begin{equation}\label{eq:predErr}
         Y-Y_{pred}=Y_{error}
\end{equation}

\end{center}
\end{frame}

%------------------------------------------------
%------------------------------------------------
\section{Evaluation Matrix}
%------------------------------------------------

\begin{frame}
\frametitle{Evaluation Matrix}
\begin{center}
Class Count $\Longrightarrow$ relative specificity\\
In Class Count $\Longrightarrow$ absolute specificity
\framebox{ \includegraphics[height=2.4in, width=4in]{figures/cpicp.eps}}


\end{center}
\end{frame}

%------------------------------------------------
%------------------------------------------------
\section{Mirror Neuron Candidates: Observation}
%------------------------------------------------

\begin{frame}
\frametitle{Mirror Neuron Candidates: Observation}
\textbf{Observation Condition}\\
\begin{center}
\framebox{ \includegraphics[height=2.5in, width=4in]{figures/nocueobs.eps}}
\end{center}

\end{frame}

%------------------------------------------------
%------------------------------------------------
\section{Mirror Neuron Candidates: Execution}
%------------------------------------------------

\begin{frame}
\frametitle{Mirror Neuron Candidates: Execution}

\textbf{Execution Condition}\\
\begin{center}
\framebox{ \includegraphics[height=2.5in, width=4in]{figures/execution.eps}}
\end{center}

\end{frame}


%------------------------------------------------
\section{Raster Plots for Unit 54}
%------------------------------------------------

\begin{frame}

\begin{center}
\frametitle{Raster Plots for Unit 54}
\framebox{ \includegraphics[height=3in, width=4in]{figures/unit54.png}}

\end{center}
\end{frame}

%------------------------------------------------

%------------------------------------------------
\section{ Observation vs. Execution Findings }
%------------------------------------------------
\begin{frame}
\frametitle{Observation vs. Execution Findings }

\textcolor{darkgreen}{\textbf{Evaluation of Observation vs. Execution Findings }}
\begin{enumerate}
\item 192 neurons are investigated and classified
\item Robust results are obtained via performing cross-validation

\item Prediction rates are significantly above chance level ($25\%$ vs. $40\%$).
\item Single and multi object decoder neurons;
\begin{itemize}

\item 54 for all objects, \textbf{general decoder}.
\item 93 for cylinder, ring and cube, \textbf{multi-object decoder}.
\item 94 and 168 for cube and cylinder,\textbf{ object specific decoder}

\end{itemize}

\end{enumerate}

\begin{block}{Contribution}
\centering Unlike existing studies in the literature mirror neuron candidates are
automatically extracted based on classification performance.\\
\centering \textcolor{darkgreen}{\textbf{the Galadriel}} 
\end{block}

\end{frame}
%------------------------------------------------

\section{Neural Representation}
%------------------------------------------------

\begin{frame}
\frametitle{Neural Representation: Cross decoding}

\centering \textcolor{darkgreen}{\textbf{Neural representation via cross-decoding}}

\begin{equation}\label{eq:linearReg2}
    X*\textcolor{red}{W}=Y
\end{equation}
\begin{center}
\framebox{ \includegraphics[height=2.in, width=3.6in]{figures/transfer.eps}}


\end{center}
\end{frame}

%------------------------------------------------
%------------------------------------------------

\section{Cross Decoding Tables}
%------------------------------------------------

\begin{frame}
\frametitle{Cross Decoding Tables}
 \textbf{$W_{obs}$ trained in Observation and tested in Execution}\\
\begin{center}
\framebox{ \includegraphics[height=1.1in, width=3in]{figures/obsw.png}}
\end{center}
\textbf{$W_{exe}$  trained in Execution and tested in Execution}\\
\begin{center}
\framebox{ \includegraphics[height=1.1in, width=3in]{figures/exew.png}}
\end{center}

\end{frame}
%------------------------------------------------
%------------------------------------------------
\section{Cross Decoding Findings }
%------------------------------------------------
\begin{frame}
\frametitle{Cross Decoding Findings }

\textcolor{darkgreen}{\textbf{Evaluation of Cross-decoding Findings }}
\begin{enumerate}
\item Weight transfer among experiment condition realized
\item Similar neural representation observed for colored neurons

\item Prediction rates are above chance level ($25\%$ vs. $70\%$-$33\%$ )
\item Population level cross decoding analyzed for;
\begin{itemize}
\item finding perfect neuron pairs for decoding
\item extracting helper neurons and best solo neurons
\end{itemize}
 
\end{enumerate}

\begin{block}{Contribution}
\centering The idea of a new term coined \textbf{Temporal Mirror Neuron} based 
on decoding characteristics of a specific neuron.
\end{block}

\end{frame}
%------------------------------------------------


%-----------------------------------------------
\begin{frame}
\frametitle{Acknowledgments}
\begin{center}

\begin{itemize}


\item Erhan Oztop
\item Eren Sezener\\
-----------------------------
\item Baris Aktemur
\item Sanem Sariel\\
-----------------------------


%\item Demis Hassabis

\item Vassilis Raos
\item Vassilis Papadourakis \\
-----------------------------

\item ...Monkey/s
\item 
The mirror neuron study was supported by the Scientific and Technological Research Council of Turkey (TUB\.{I}TAK) with project number 113S391. 
\end{itemize}


\end{center}

\end{frame}

%------------------------------------------------
\section{ Publication}
%------------------------------------------------

\begin{frame}

\begin{center}
\frametitle{Publications}
\centering \textbf{Computational Neuroscience (CNS'15)}\\
\framebox{ \includegraphics[height=0.7in, width=2.5in]{figures/cns.jpg}}\\
\centering \textbf{National Neuroscience (USK'15)}\\
\framebox{ \includegraphics[height=0.7in, width=2.5in]{figures/usk.png}}\\
\centering \textbf{Humanoids'13}\\
\framebox{ \includegraphics[height=0.7in, width=2.5in]{figures/humanoid.png}}

\end{center}
\end{frame}
%------------------------------------------------


%-----------------------------------------------
%\begin{frame}
%\frametitle{Quotation}
%\begin{center}
%
%
%\huge \textcolor{black}{``If you see fraud and don't shout fraud, you are a fraud"} 
%\small Nassim N. Taleb
%\end{center}
%
%\end{frame}

%-----------------------------------------------
\begin{frame}
\frametitle{Thanks}
\begin{center}


\huge \textcolor{darkgreen}{$</Thanks>$} 
\end{center}

\end{frame}



\end{document} 
